\documentclass[sigconf]{acmart}

\usepackage{hyperref}

\usepackage{endfloat}
\renewcommand{\efloatseparator}{\mbox{}} % no new page between figures

\usepackage{booktabs} % For formal tables

\settopmatter{printacmref=false} % Removes citation information below abstract
\renewcommand\footnotetextcopyrightpermission[1]{} % removes footnote with conference information in first column
\pagestyle{plain} % removes running headers

\begin{document}
\title{Big data in Clinical Trials}


\author{Mohan Mahendrakar}
\orcid{1234-5678-9012}
\affiliation{%
  \institution{Indiana University}
  \streetaddress{P.O. Box 1212}
  \city{Bloomington} 
  \state{Indiana} 
  \postcode{43017-6221}
}
\email{mmahendr@iu.edu}

% The default list of authors is too long for headers}
\renewcommand{\shortauthors}{B. Trovato et al.}


\begin{abstract}
This paper will help us to understand about Clinical Trials and how Big
data is impacting Clinical Trials. Oncology (Trials) is undergoing a
data-driven metamorphosis. Armed with new and ever more efficient
molecular and information technologies, we have entered an era where big
data is helping us spearhead the fight against various deceases. This
technology driven data explosion, often referred to as "big data".
\cite{TR01}
\end{abstract}

\keywords{I523, HID 326, Big data, Clinical, Trials, Health care, Data
integration, Analytics}

\maketitle

\section{Introduction}
A primary objective of clinical trials is gaining knowledge from studying
a subset of patients which can then be applied to a much wider group of
patients to improve care. In routine practice, patient care is delivered
within a rich background of intrinsic and endemic confounding factors and
biases associated with practices and patients.  \cite{TR02}

The data collected around the world from various patients, deceases form
big data (collection of large data sets). Big data is currently being
used on a limited basis in the clinical trials arena, but experts believe
its widespread use is coming in the near future. Some hail the great
promise it holds in furthering drug discovery. Others are skeptical that
it will bring much value and say that enthusiasm should be tempered. \cite{TR05} 

%%%%%%%%%%%%%%%%%%%%%%%%%%%%%%%%%%%%%%%%%%%%%%%%%%%%%%%%%%%%%%%%%%%
According to IBM, 2.3 trillion gigabytes of data are created every 
day—so much that 90\% of the data in the world today has been created
in the last two years alone. Digital Universe estimates that by 2020,
there will be 5,200 gigabytes of data for every man, woman and child 
on Earth.

It is predicted that the market for Big Data technology and services
will reach \$16.9 billion in 2015, up from \$3.2 billion in 2010. 
This is an annual growth rate of 40 percent, which is about seven 
times the rate of the overall information and communications 
technology market. According to CB insights, health care investments
in Big Data totaled \$274.5 million in 2012, and it went to \$371.5 
million in 2013. \cite{TR03}

%%%%%%%%%%%%%%%%%%%%%%%%%%%%%%%%%%%%%%%%%%%%%%%%%%%%%%%%%%%%%%%%%%%
\section{Big Data \& clinical research}
Discovering clinical trials hidden patterns and associations within
the heterogeneous data, uncovering new bio markers and drug targets.
Allowing the development of predictive disease progression models.
Analyzing Real World Data (RWD) as a complementary instrument to 
clinical trials, for the rapid development of new personalized 
medicines. The development of advanced statistical methods for 
learning causal relations from large scale observational data is a
crucial element for this analysis. \cite{TR04}

\subsection{Data Integration}
%%%%%%%%%%%%%%%%%%%%%%%%%%%%%%%%%%%%%%%%%%%%%%%%%%%%%%%%%%%%%%%%%%%
Having access to consistent, reliable, and well linked is one 
of the biggest challenges facing pharmaceutical clinical trials. The
ability to manage and integrate data generated at all phases of the
value chain, from discovery to real-world use after regulatory 
approval, is a fundamental requirement to allow companies to derive
maximum benefit from the technology trends. Data are the foundation
upon which the value-adding analytics are built. Effective end-to-end
data integration establishes an authoritative source for all pieces of
information and accurately links disparate data regardless of the 
source be it internal or external, proprietary or publicly available.
Data integration also enables comprehensive searches for subsets of
data based on the linkages established rather than on the information
itself. "Smart" algorithms linking laboratory and clinical data, for
example, could create automatic reports that identify related 
applications or compounds and raise red flags concerning safety or 
efficacy. \cite{TR02}

%%%%%%%%%%%%%%%%%%%%%%%%%%%%%%%%%%%%%%%%%%%%%%%%%%%%%%%%%%%%%%%%%%%
Implementing end-to-end data integration requires a number of
capabilities, including trusted sources of data and documents, the
ability to establish cross-linkages between elements, robust quality
assurance, workflow management, and role-based access to ensure that
specific data elements are visible only to those who are authorized
to see it. Pharmaceutical companies generally avoid overhauling their
entire data-integration system at once because of the logistical
challenges and costs involved, although at least one global 
pharmaceutical enterprise has employed a "big bang" approach to 
remaking its clinical IT systems. \cite{TR02}

%%%%%%%%%%%%%%%%%%%%%%%%%%%%%%%%%%%%%%%%%%%%%%%%%%%%%%%%%%%%%%%%%%%
Data is being generated by different sources and comes in a variety
of formats including unstructured data. All of this data needs to 
be integrated or ingested into Big Data Repositories or Data 
Warehouses. This involves at least three steps, namely, Extract, 
Transform and Load (ETL). With the ETL processes that have to be 
tailored for medical data have to identify and overcome structural,
syntactic, and semantic heterogeneity across the different data 
sources. The syntactic heterogeneity appears in forms of different 
data access interfaces, which were mentioned above, and need to be
wrapped and mediated. Structural heterogeneity refers to different
data models and different data schema models that require 
integration on schema level. Finally, the process of integration 
can result in duplication of data that requires consolidation.

%%%%%%%%%%%%%%%%%%%%%%%%%%%%%%%%%%%%%%%%%%%%%%%%%%%%%%%%%%%%%%%%%%%
The process of data integration can be further enhanced with 
information extraction, machine learning, and semantic web 
technologies that enable context based information interpretation.
Information extraction will be a mean to obtain data from additional
sources for enrichment, which improves the accuracy of data 
integration routines, such as duplication and data alignment. 
Applying an active learning approach ensures that the deployment of
automatic data integration routines will meet a required level of 
data quality. Finally, the semantic web technology can be used to 
generate graph based knowledge bases and oncologies to represent 
important concepts and mappings in the data. The use of standardized 
oncologies will facilitate collaboration, sharing, modelling, and 
reuse across applications. \cite{TR04}

%%%%%%%%%%%%%%%%%%%%%%%%%%%%%%%%%%%%%%%%%%%%%%%%%%%%%%%%%%%%%%%%%%%
\subsection{Exascale computing}
After data integration is completed, the big question is how to process
such huge volume of the data? There will be use cases, e.g. precision
medicine, where the promises brought by Big Data will only be fulfilled
through dramatic improvements in computational performance and capacity,
along with advances in software, tools, and algorithms. Exascale
computers-machines that perform one billion calculations per second and
are over 100 times more powerful than today's fastest systems will be
needed to analyses vast stores of clinical and genomic data and develop
predictive treatments based on advanced 3D multi-scale simulations with
uncertainty quantification. Precision medicine will also require scaling
these systems down, so clinicians can incorporate research breakthroughs
into everyday practice. \cite{TR04}
%%%%%%%%%%%%%%%%%%%%%%%%%%%%%%%%%%%%%%%%%%%%%%%%%%%%%%%%%%%%%%%%%%%

\subsection{Data-driven metamorphosis}
Data collected in clinical trials undergoing a data-driven 
metamorphosis. Armed with new and ever more efficient molecular 
and information technologies, we have entered an era where data is
helping us spearhead the fight against cancer. This technology 
driven data explosion, often referred to as "big data", is not only
expediting biomedical discovery, but it is also rapidly transforming
the practice of oncology into an information science. This evolution
is critical, as results to-date have revealed the immense complexity
and genetic heterogeneity of patients and their tumors, a sobering
reminder of the challenge facing every patient and their oncologist
. This can only be addressed through development of clinico-molecular
data analytics that provide a deeper understanding of the mechanisms
controlling the biological and clinical response to available 
therapeutic options. Beyond the exciting implications for improved 
patient care, such advancements in predictive and evidence-based 
analytics stand to profoundly affect the processes of cancer drug
discovery and associated clinical trials. \cite{TR01}
%%%%%%%%%%%%%%%%%%%%%%%%%%%%%%%%%%%%%%%%%%%%%%%%%%%%%%%%%%%%%%%%%%%
\subsection{Big data analytics}
Medical research has always been a data-driven science, with 
randomized clinical trials being a gold standard in many cases. 
However, due to recent advances in omics-technologies, medical 
imaging, comprehensive electronic health records, and smart devices,
medical research as well as clinical practice are quickly changing 
into Big Data-driven fields. As such, the healthcare domain as a 
whole - doctors, patients, management, insurance, and politics -
can significantly profit from current advances in Big Data 
technologies, and from analytics. \cite{TR04}
%%%%%%%%%%%%%%%%%%%%%%%%%%%%%%%%%%%%%%%%%%%%%%%%%%%%%%%%%%%%%%%%%%%
\subsection{Machine Learning}
Many healthcare applications would significantly benefit from the
processing and analysis of multimodal data - such as images, 
signals, video, 3D models, genomic sequences, reports, etc. 
Advanced machine learning systems can be used to learn and relate
information from multiple sources and identify hidden correlations 
not visible when considering only one source of data. For instance,
combining features from images (e.g. CT scans, radiographs) and text
(e.g. clinical reports) can significantly improve the performance 
of solutions. \cite{TR04}
%%%%%%%%%%%%%%%%%%%%%%%%%%%%%%%%%%%%%%%%%%%%%%%%%%%%%%%%%%%%%%%%%%%

\section{Challenges}
Big pharma companies typically keep their cards close to the vest 
because it costs so much to develop a drug throughout its lifetime.
From discovery to prescription pad, a typical medication can take 
twelve years and \$4 billion to shepherd through its lifecycle, a
significant investment that would be hard to recoup if everyone had
the secret to the newest blockbuster pill, especially since only ten
percent of drugs ever make it to market. \cite{AR06}

Although there is already a 
huge amount of healthcare data around the world and while it is 
growing at an exponential rate, nearly all the data is stored in
individually. Data collected by a clinic or by a hospital is mostly 
kept within the boundaries of the healthcare provider. Moreover, 
data stored within a hospital is hardly ever integrated across 
multiple IT systems. For example, if we consider all the available 
data at a hospital from a single patient’s perspective, information 
about the patient will exist in the EMR system, laboratory, imaging 
system and prescription databases. Information describing which 
doctors and nurses attended to the specific patient will also exist.
However, in most of cases, every data source mentioned here is stored
in separate silos. Thus, deriving insights and therefore value from 
the aggregation of these data sets is not possible at this stage. It
is also important to realize that in today's world a patient's 
medical data does not only reside within the boundaries of a 
healthcare provider. The medical insurance and pharmaceuticals 
industries also hold information about specific claims and the
characteristics of prescribed drugs respectively. Increasingly,
patient-generated data from IoT devices such as fitness trackers, 
blood pressure monitors and weighing scales are also providing 
critical information about the day-to-day lifestyle
characteristics of an individual. Insights derived from such data
generated by the linking among EMR data, vital data, laboratory
data, medication information, symptoms (to mention some of these)
and their aggregation, even more with doctor notes, patient
discharge letters, patient diaries, medical publications, namely
linking structured with unstructured data, can be crucial to
design coaching programs that would help improve people's
lifestyles and eventually reduce incidences of chronic disease,
medication and hospitalization. \cite{TR04}
%%%%%%%%%%%%%%%%%%%%%%%%%%%%%%%%%%%%%%%%%%%%%%%%%%%%%%%%%%%%%%%%%%%
\section{conclusion}
%%%%%%%%%%%%%%%%%%%%%%%%%%%%%%%%%%%%%%%%%%%%%%%%%%%%%%%%%%%%%%%%%%%g
The recent surge in big data initiatives in health care is expected 
to have a positive impact on clinical trials. Increased 
standardization of common data elements and nomenclature should 
assist in streamlined trial design and exchange of data. Standardize
between trials and will allow easier multi-study analysis. 
Standardization and quality improvement efforts go hand in hand with
a maturing big data infrastructure providing collateral benefits to
data curation for trials. \cite{TR01}
%%%%%%%%%%%%%%%%%%%%%%%%%%%%%%%%%%%%%%%%%%%%%%%%%%%%%%%%%%%%%%%%%%%
\begin{acks}

The authors would like to thank to Professor and TAs for guiding in
making the better paper.

\end{acks}

\bibliographystyle{ACM-Reference-Format}
\bibliography{report} 

\end{document}
